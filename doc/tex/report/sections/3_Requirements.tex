\newpage
\section{Requisiti}
\subsection{Business}
Il sistema dovrà emulare il gioco Plants Vs Zombies.
Sono stati individuati i seguenti requisiti di business:
\begin{itemize}
    \item Il gioco dovrà avere un menù di inizio e fine partita.
    \item Sarà possibile giocare una partita intera ad un clone di Plant Vs Zombies.
    \item Sarà possibile visualizzare la mappa di gioco e interagire con essa.
\end{itemize}

\subsection{Funzionali}
Il gioco si compone di un insieme di entità e regole, per quanto riguarda gli elementi della partita:
\begin{itemize}
    \item Le entità in gioco possono essere di tre tipi:
    \begin{itemize}
        \item Zombie
        \item Piante
        \item Bullet
    \end{itemize}
    \item Il campo da gioco deve essere composto da più corsie.
    \item Il gioco sarà composto da tre schermate: menù iniziale, schermata di gioco e menù finale.
    \item I nemici avanzeranno lungo le corsie da destra verso sinistra.
    \item Ogni Zombie e ogni Pianta deve avere un campo visivo.
    \item Gli Zombie saranno organizzati in ondate.
    \item Le ondate contengono una serie di Zombie, la cui dimensione e complessità aumenta con l'avanzare del gioco.
    \item Quando uno Zombie si avvicina sufficientemente ad una Pianta, lo Zombie la attaccherà.
    \item Quando uno Zombie entra nel campo visivo di una Pianta, questa lo attaccherà.
    \item Se una Pianta o uno Zombie viene colpito, perderà vita.
    \item Se una Pianta o uno Zombie rimane senza vita, morirà e verra rimossa dal campo di gioco.
    \item Se uno Zombie raggiunge la fine della corsia, la partita termina.
    \item Ogni Pianta avrà un costo di piazzamento.
    \item Per piazzare una Pianta è necessario avere un numero di risorse maggiore o uguale al costo della Pianta.
    \item I tipi di Piante previsti sono:
    \begin{itemize}
        \item Peashooter: spara Bullet che colpiscono il primo Zombie che incontrano.
    \end{itemize}
    \item I tipi di zombie previsti sono:
    \begin{itemize}
        \item BasicZombie: quando in contra una pianta, la attacca.
    \end{itemize}
    \item I tipi di Bullet previsti sono:
    \begin{itemize}
        \item Peabullet: il Bullet sparato dal Peashooter.
        \item PawBullet: il Bullet sparato dal BasicZombie.
    \end{itemize}
\end{itemize}

\subsubsection{Utente}
L'utente deve poter:
\begin{itemize}
    \item Avviare una partita.
    \item Piazzare le Piante nelle celle libere del campo da gioco.
    \item Vedere le statistiche a fine partita e riavvare il gioco.
\end{itemize}

\subsubsection{Non Funzionali}
I requisiti non funzionali individuati per il progetto sono:
\begin{itemize}
    \item Realizzazione di software estendibile e rivisitabile.
    \item Realizzazione di un'esperienza di gioco godibile (usabilità, bilanciamento, qualità grafica).
    \item Il gioco deve rimanere fluido per un numero ragionevole di ondate.
\end{itemize}

\subsection{Opzionali}
\begin{enumerate}
    \item Possibilità di mettere in pausa, riprendere e velocizzare il gioco.
    \item Aumentare i tipi di Piante.
    \item Aumentare i tipi di Zombie.
    \item Aumentare i tipi di Bullet.
    \item Implementazione di un meccanismo di gestione della difficoltà.
    \item Inserire delle animazioni.
    \item Inserire la possibilità di generare campi di gioco particolari.
\end{enumerate}

\subsection{Implementativi}
Il gioco verrà sviluppato in Scala e dipenderà da alcuni framework aggiuntivi quali libGDX, Akka e TuProlog.
Il software prodotto deve essere testato con ScalaTest per garantire la manutenzione, la qualità e la corretta integrazione del codice dei vari componenti del team.