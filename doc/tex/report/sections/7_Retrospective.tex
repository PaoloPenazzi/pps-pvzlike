\newpage
\section{Retrospettiva}
I vari documenti relativi alle singole iterazioni (sprint), inclusi quelli di planning e retrospettiva, sono reperibili all'ultima release nel file backlog.pdf.


In questa ultima sezione vogliamo discutere a posteriori del processo di sviluppo, della sua evoluzione, e dei lati positivi e negativi riscontrati.

\subsection{Processo di sviluppo}
\subsubsection{Daily Meeting}
Durante il primo e secondo sprint abbiamo adottato un processo che includeva daily meeting. Durante queste riunioni veniva anche redatto un verbale. Dal terzo sprint a causa della difficoltà dei membri del team nell'accordarsi per un orario fisso abbiamo gradualmente rimosso questo appuntamento. Essere riusciti a mantenere inizialmente i daily sprint ha facilitato il processo di design incrementale. Lo svantaggio della loro rimozione è stata una maggiore difficoltà nel mantenere l'allineamento dei membri del team su di una visione d'insieme del processo di sviluppo.

\subsubsection{Pull request vs Pair programming}
Utilizzando il meccanismo delle pull request, ci siamo resi conto di quanto questo sia utile per mantenere traccia di tutte le decisioni e per favorire il lavoro asincrono.
La tendenza del gruppo è stata quella di cooperare prevalentemente tramite chiamata su Teams durante lo sviluppo delle feature. In questi casi si è perso il vantaggio delle pull request, ma si è ottenuta una maggiore agilità.
Crediamo che il team abbia raggiunto il giusto compromesso da questo punto di vista, utilizzando il meccanismo delle pull request per le feature più corpose e significative, e lasciando alla collaborazione vocale/pair programming il resto.

Altro vantaggio dell'ampia collaborazione vocale non citato: sopperire alla distanza fisica del team (che ha lavorato diviso tra Italia e Svezia).

\subsubsection{Tools}
Il processo di sviluppo avrebbe giovato di una maggiore integrazione tra product backlog e Trello. L'elemento di viscosità riscontrato è stato il dover riscrivere i task con relativa descrizione in entrambi i posti.



\subsection{Commenti finali}
Il sentimento comune di tutti i componenti del gruppo è che Scala si adatta molto bene ad una metodologia di sviluppo Agile, grazie all'enorme flessibilità derivante dall'arsenale di meccanismi che mette a disposizione.
