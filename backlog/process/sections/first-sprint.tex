\section{First Sprint Planning - 07/09/2022}
Il team si è incontrato via Microsoft Teams per rivedere le specifiche del progetto e definire i task del primo sprint.\\
L'incontro ha avuto una durata di 3 ore.

\subsection{Scelte ulteriori riguardo alle tecnologie}
Il team ha valutato diverse opzioni per la componente grafica del progetto, e le principali sono:
\item ScalaFX, è la più utilizzata e con più materiale associato;
\item Indigo, specializzata per la creazione di giochi.

Indigo consiste in un vero e proprio game engine, e ci darebbe meno flessibilità nelle scelte architetturali, perdendo un po' il goal principale di questo progetto/corso.\\

Il team ha poi deciso di utilizzare Akka per implementare una gestione ad attori del sistema. Akka è un toolkit ormai maturo e che pensiamo ci possa portare beneficio
durante le fasi di analisi e di sviluppo del progetto. Inoltre sfruttiamo le conoscenze acquisite su Akka durante il corso di Programmazione Concorrente e Distribuita.

\subsection{Obiettivi}
\begin{itemize}
  \item Setup del progetto con tecnologie selezionate (Github, Github actions, Trello, Sbt, Intellij IDEA);
  \item Creazione del model basilare associato ad una torretta;
  \item Creazione del model basilare associato ad un nemico;
  \item Creazione della griglia di gioco in ScalaFX;
  \item Creazione di un controller/game loop.
\end{itemize}

\subsection{Planning}
Incontro giornaliero (durata 10-15 minuti l'uno, si terrà la mattina).
Deadline sprint: 15 settembre.

\subsubsection{Suddivisione del lavoro}
\begin{itemize}
  \item Alpi =\textgreater  Creazione della griglia di gioco in ScalaFX;
  \item Foschini =\textgreater  Creazione del model basilare associato ad un nemico;
  \item Parrinello =\textgreater  Creazione di un controller/game loop;
  \item Penazzi =\textgreater  Creazione del model basilare associato ad una torretta;
  \item All =\textgreater  Setup del progetto con tecnologie selezionate, definire il tema di gioco.
\end{itemize}

\subsubsection{Sprint Review - 18/09/2022}
Risultati:
\begin{itemize}
  \item Il setup del progetto è stato effettuato.
  \item Il model basilare associato ad una torretta è stato creato.
  \item Il model basilare associato ai nemici è stato creato.
\end{itemize}

Nel setup non è stata creato il workflow per la continous delivery, al momento non è necessario e il gruppo ha preferito concentrarsi sul altri aspetti.
Per cause di forza maggiore Alpi e Parrinello non sono riusciti a dedicare molto tempo al progetto e quindi non sono riusciti a portare a termine i propri task. Gli altri membri del gruppo erano al corrente della situazione e avevano già dato dichiarato che non sarebbe stato un problema.
Il processo definito inizialmente è migliorabile, in particolare per quanto riguarda i daily meeting: a causa dei molteplici impegni di ogni componente ci siamo resi conto che è più funzionale effettuare il "daily meeting" ogni due giorni.
Al termine del primo sprint, come previsto, non è stata effettuata alcuna release.
