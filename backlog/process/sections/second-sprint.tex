\section{Second Sprint Planning - 18/09/2022}
Il team si è incontrato via Microsoft Teams per definire gli obiettivi del secondo sprint.

\subsection{Obiettivi}
\begin{itemize}
  \item Creazione della griglia di gioco in ScalaFX.
  \item Creazione ad alto livello un controller/game loop.
  \item Creazione degli attori responsabili del controllo delle entità del gioco (torrette, nemici, etc..).
  \item Renderizzazione della griglia e delle entita in gioco (seppur immobili).
  \item Creare una prima versione semplificata del gioco.
\end{itemize}

\subsection{Planning}
Stand-up meetting ogni due giorni (durata 15-20 minuti l'uno).
Deadline sprint: 3 Ottobre.

\subsubsection{Suddivisione del lavoro}
\begin{itemize}
  \item Alpi =\textgreater  Creazione della griglia di gioco in ScalaFX;
  \item Parrinello =\textgreater  Creazione ad alto livello un controller/game loop.
  \item Foschini, Penazzi =\textgreater  Creazione degli attori responsabili del controllo delle entità del gioco (torrette, nemici, etc..).
  \item Alpi, Penazzi =\textgreater  Renderizzazione della griglia e delle entita in gioco (seppur immobili).
  \item All =\textgreater  Creare una prima versione semplificata del gioco.
\end{itemize}

\subsubsection{Sprint Review - 07/10/2022}
Risultati:
\begin{itemize}
  \item La parte di renderizzazione delle entità è stata realizzata.
  \item Una prima versione del controller e del game loop sono stati implementati.
  \item È stato creato un attore per ogni entità del gioco.
  \item È stata effettuata una prima release che mostra una demo del gioco.
  \item È stata create una prima versione del controller delle ondate.
  \item Il model delle entità è stato ulteriormente sviluppato.
\end{itemize}

Si è deciso di cambiare il framework che si occupa della grafica: ScalaFX è stato sostituito con LibGDX.
A causa delle tante ore di lezione in momenti diversi della giornata non sono stati effettuati dei veri e propri stand-up meeting.
Il gruppo ha comunque lavorato a stretto contatto (con frequenti videochiamate) ed è riuscito a coordinarsi bene.
Foschini ha contribuito in maniera minore per questo sprint perchè assente per una settimana, ma era stato preventivato.
La release è arrivata con due giorni di ritardo perchè il gruppo ha dovuto spendere tempo non preventivato nel raffinamento del design dell'architettura.
Per il prossimo sprint, si cercherà di fare stime più accurate.