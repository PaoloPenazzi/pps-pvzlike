\section*{Third Sprint Planning - 07/10/2022}
Il team si è incontrato via Microsoft Teams per definire gli obiettivi del terzo sprint.

\subsection*{Obiettivi}
\begin{itemize}
    \item Gestione delle collisioni tra entità.
    \item Renderizzazione del campo di gioco.
    \item Modellazione della valuta di gioco per piazzare le torrette.
    \item Modellazione delle ondate di gioco.
    \item Schermata di gioco con interfaccia completa.
    \item Raffinamento del modello dei nemici e degli attori relativi.
    \item Modellazione dello stato delle entità della partita.
    \item Rendere il gioco più leggero in termini di utilizzo di risorse hw.
    \item Generazione delle wave con Prolog. (Opzionale)
    \item Realizzazione di una versione del gioco che permetta di giocare una wave.
\end{itemize}

\subsection*{Planning}
Deadline sprint: 22 Ottobre.

\subsubsection*{Suddivisione del lavoro}
\begin{itemize}
    \item Parrinello => Gestione delle collisioni tra entità.
    \item All => Renderizzazione del campo di gioco.
    \item Parrinello => Modellazione della valuta di gioco per piazzare le torrette.
    \item Penazzi => Modellazione delle ondate di gioco.
    \item Foschini => Schermata di gioco con interfaccia completa.
    \item Foschini => Raffinamento del modello dei nemici e degli attori relativi.
    \item Alpi => Modellazione dello stato delle entità della partita.
    \item Alpi => Rendere il gioco più leggero in termini di utilizzo di risorse hw.
    \item Penazzi, Parrinello => Generazione delle wave con Prolog. (Opzionale)
    \item All => Realizzazione di una versione del gioco che permetta di giocare una wave.
\end{itemize}

